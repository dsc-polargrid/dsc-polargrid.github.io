@proceedings {7966,
	title = {Building the PolarGrid Portal Using Web 2.0 and OpenSocial},
	journal = {The International Conference for High Performance Computing, Networking, Storage and Analysis (SC{\textquoteright}09)},
	year = {2009},
	month = {11/20/2009},
	pages = {5},
	publisher = {ACM Press},
	chapter = {Session: Social Networking and Gateways},
	address = {Porland, OR},
	abstract = {Science requires collaboration. In this paper, we investigate the feasibility of coupling current social networking techniques to science gateways to provide a scientific collaboration model. We are particularly interested in the integration of local and third party services, since we believe the latter provide more long-term sustainability than gateway-provided service instances alone. Our prototype use case for this study is the PolarGrid portal, in which we combine typical science portal functionality with widely used collaboration tools. Our goal is to determine the feasibility of rapidly developing a collaborative science gateway that incorporates third-party collaborative services with more typical science gateway capabilities. We specifically investigate Google Gadget, OpenSocial, and related standards.},
	keywords = {Collaboration tools, Gadget, OAuth, OpenID, OpenSocial, REST, Web 2.0},
	isbn = {978-1-60558-887-2 },
	doi = {http://doi.acm.org/10.1145/1658260.1658267},
	url = {http://grids.ucs.indiana.edu/ptliupages/publications/GCE09_PolarGridPortal_Guo_Singh_Pierce_PostAccept_Final.pdf},
	author = {Zhenhua (Gerald) Guo and Raminderjeet Singh and Pierce, Marlon}
}
